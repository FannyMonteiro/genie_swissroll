% cGENIE Examples document

% Andy Ridgwell, March 2011
%
% ---------------------------------------------------------------------------------------------------------------------------------
% 11/03/24: added CH4 'wetland' example
% 11/05/25: added preindustrial example
% ---------------------------------------------------------------------------------------------------------------------------------

\documentclass[10pt,twoside]{article}
\usepackage[paper=a4paper,portrait=true,margin=2.5cm,voffset=0pt,ignorehead,footnotesep=1cm]{geometry}
\usepackage{graphicx}
\usepackage{hyperref}
\usepackage{paralist}
\usepackage{caption}
\usepackage{float}

\linespread{1.1}
\setlength{\pltopsep}{2.5pt}
\setlength{\plparsep}{2.5pt}
\setlength{\partopsep}{2.5pt}
\setlength{\parskip}{2.5pt}

\title{cGENIE Examples}
\author{Andy Ridgwell}
\date{\today}

\begin{document}


%=================================================================================================================================
%=== BEGIN DOCUMENT ==============================================================================================================
%=================================================================================================================================

\maketitle


%=================================================================================================================================
%=== CONTENTS ====================================================================================================================
%=================================================================================================================================

\tableofcontents



%=================================================================================================================================
%=== CHAPTERS ====================================================================================================================
%=================================================================================================================================


%---------------------------------------------------------------------------------------------------------------------------------
%--- Introduction ----------------------------------------------------------------------------------------------------------------
%---------------------------------------------------------------------------------------------------------------------------------

\newpage
\section{Introduction}\label{Introduction}

The document is intended to provide practical help in configuring and running experiments using \texttt{cGENIE}.
\\The first section describes a number of different \texttt{cGENIE} configurations their spin-ups. 
\\The second section contains a variety of 'illustrative' exercises using these spin-ups.
\\Note that this document should be read in conjunction with the \texttt{cGENIE} \textit{HOW-TO}, which contains related and supporting information.


%---------------------------------------------------------------------------------------------------------------------------------
%--- Example cGENIE configurations and spin-ups ----------------------------------------------------------------------------------
%---------------------------------------------------------------------------------------------------------------------------------

\newpage
\section{Example configurations and spin-ups}\label{Example configurations and spin-ups}

A variety of different model configurations and \textit{spin-up} designs have been created for reference and for use as a helpful starting-point (template) in creating model experiments.
Briefly: the \texttt{cGENIE} example configurations and \textit{spin-ups} consist of 3 components (see the \textit{User manual} for a full description):

\begin{compactenum}

	\item A \textit{base-config} file, which contains the parameter values defining the basic grid dimensions (and continental configuration, the number of \textit{tracers}, and the basic physics configuration.
	
	\item A \textit{user-config} file containing the parameter value changes for a specific experiment.
	
	\item A set of files defining any selected forcings of biogeochemical tracers. The tracer \textit{forcing} definitions are provided in the form of subdirectories located in \texttt{\~{}/cgenie/genie-forcings}, with each containing:
	
	\begin{compactitem}
		\item 3 files defining which \textit{forcings} are to be selected, what sort of \textit{forcing} is required (i.e., \textit{restoring} vs. \textit{flux}), and any additional information, with a filename of the format:
		\\ (\texttt{configure\_forcings\_xxx.dat}).
		\item A series of forcing definition files (\texttt{biogem\_force\_*.dat}) for each biogeochemical forcing selected (if any).
	\end{compactitem}
	
	The names of the subdirectories correspond to the value of the \texttt{bg\_par\_fordir\_name} namelist parameter set in the user configuration file. If the \texttt{bg\_par\_fordir\_name} namelist parameter is not set in the user config file then the default setting is used (no tracer forcings are selected).
	
\end{compactenum}

\noindent The various example configurations and spin-ups provided are\footnote{Unless otherwise stated, there is \textbf{no} feedback between CO2 and climate enabled in the experimental designs as provided, i.e., any simulated change in atmospheric CO2 will not affect climate. This can be changed by setting: \texttt{ea\_36=y}. Radiative forcing of the EMBM atmospheric module will then follow the relative (log) deviation (from 278 ppm) of CO2.}\footnote{Unless otherwise stated, a 'CO2-calcification' feedback (see: \textit{Ridgwell et al.} [2007a,b] is enabled by default, with marine (pelagic) CaCO3 production is calculated based on ambient environmental (saturation) conditions. In order to fix the CaCO3:POC rain ratio (either spatially uniform, or with a (pre-calculated) spacial pattern -- see the \textit{HOW-TO}.}\footnote{Only a selection of variables will be saved in the form of the spatial data fields (2- and 3-D) and data as \textit{time-series}. Not all the predicted results that you want might will therefore necessarily be saved. (Alternatively, rather more data than you care to look at might be saved (thus bloating the netCDF fields) ... ). Either way, you can adjust the data that is saved by changing existing or adding new parameter specifications in the \textit{user-config} file. Refer to the \textit{User manual} for more information.}\footnote{Similarly -- data may not be saved with a sufficient frequency (or alternatively too frequently) or simply not at the required points in time.}\footnote{Each experiment specifies a file containing a list of time-slice years (at which 2-D and 3-D fields will be saved) pointed to by the parameter \texttt{bg\_par\_infile\_slice\_name}. The file containing the time-series years (at which time-series data is saved) is pointed to by the namelist parameter \texttt{bg\_par\_infile\_sig\_name}. To change the frequency and/or timing of data saving for time-slice or time-series data saving, either edit one of the files provided on SVN (some of which are used in the \textit{user-configs} provided), or create a new file and set the relevant namelist parameter equal its name. Again -- refer to the \textit{User manual} for full details.}:

\begin{compactenum}

\item \texttt{EXAMPLE\_worjh2\_PO4\_SPIN} -- The most basic, ocean-only [16-level, seasonal ocean] carbon (+ 13C) cycle including P only limitation of marine productivity. Pre-industrial boundary conditions.
\item \texttt{EXAMPLE\_worjh2\_PO4\_PREINDUST} -- A basic ocean-only [16-level, seasonal ocean] carbon (+ 13C) cycle with P of marine productivity only, PLUS radiocarbon and CFC tracers. Pre-industrial boundary conditions.
\item \texttt{EXAMPLE\_worjh2\_PO4Fe\_SPIN} -- A basic ocean-only [16-level, seasonal ocean] carbon (+ 13C) cycle including P + iron co-limitation of marine productivity. Pre-industrial boundary conditions.
\item \texttt{EXAMPLE\_worjh2\_PO4Fe\_CH4\_SPIN} -- A basic ocean-only [16-level, seasonal ocean] carbon (+ 13C) cycle including P + iron co-limitation of marine productivity, PLUS a CH4 (+ 13C) cycle. Pre-industrial boundary conditions.
\item \texttt{EXAMPLE\_p0055c\_PO4\_CH4\_SPIN} -- A basic ocean-only [16-level, seasonal ocean] carbon (+ 13C) cycle with P limitation of marine productivity only, PLUS a CH4 (+ 13C) cycle. Early Eocene boundary conditions.
\item \texttt{EXAMPLE\_p0055c\_PO4\_CH4\_SPIN2} -- A basic ocean-only [16-level, seasonal ocean] carbon (+ 13C) cycle including P + iron co-limitation of marine productivity, PLUS a CH4 (+ 13C) cycle configured as an 'open' CH4 cycle. Early Eocene boundary conditions.
\item \texttt{EXAMPLE\_p0055c\_PO4\_CH4\_S72x72\_SPIN} -- A basic ocean-only [16-level, seasonal ocean] carbon (+ 13C) cycle including P limitation of marine productivity only, PLUS deepsea sediments (on a 72x27 grid) and weathering, configured as a 'closed' system. Early Eocene boundary conditions.
\item \texttt{EXAMPLE\_p0055c\_PO4\_CH4\_S72x72\_SPIN2} -- A basic ocean-only [16-level, seasonal ocean] carbon (+ 13C) cycle including P limitation of marine productivity only, PLUS deepsea sediments (on a 72x27 grid) and weathering, configured as a 'closed' system. Early Eocene boundary conditions.

\end{compactenum}

\noindent The syntax for the command-line launching of a model experiment is as per detailed in the \textit{User manual}, i.e.:
\vspace{-5pt}\begin{verbatim}./runcgenie.sh cgenie_eb_go_gs_ac_bg_itfclsd_16l_JH_BASE /
EXAMPLE_worjh2_PO4_SPIN 10000
\end{verbatim}\vspace{-5pt}
this is all on ONE LINE (although in practice it may wrap on a normal screen width), and the components must be SPACE SEPERATED.
\\Again: \textbf{ONE LINE; SPACE SEPERATED}

\noindent You can use the example experimental configurations provided as a template for your own experiments. To do this, just copy the user config file and rename it. Alter the namelist values contained in it and/or add additional parameter changes from the defaults to the end of the file. Forcing directories can be similarly copied and edited, or they could be used unaltered for a variety of experiments.\footnote{The \textit{user-config} files provided may have to be edited consistent with your local software environment (particularly with how the home directory is defined/represented).}
	

%---------------------------------------------------------------------------------------------------------------------------------

\subsection{Modern 36x36x16 configuration; preindustrial}\label{EXAMPLE_worjh2_PO4_SPIN}

The base \textit{spin-up} configuration.

\noindent \textbf{Physics configuration}: GOLDSTEIN ocean + sea-ice + EMBM atmosphere modules. Climatology is seasonal and identical to that described in \textit{Cao et al.} [2009] (and references therein).

\noindent \textbf{Biogeochemistry configuration}: Basic ocean (and atmosphere) carbon cycle as described \textit{Cao et al.} [2009]. Atmospheric restoring of CO2 (plus d13C).

\noindent \textbf{Base-config} The \textit{base-config} file is named:
\vspace{-10pt}\begin{verbatim}cgenie_eb_go_gs_ac_bg_itfclsd_16l_JH_BASE\end{verbatim}\vspace{-10pt}

\noindent \textbf{User-config} The \textit{user-config} file is named\footnote{The model experiment will be assigned the same name as this when using \texttt{runcgenie.sh}.}:
\vspace{-10pt}\begin{verbatim}EXAMPLE_worjh2_PO4_SPIN\end{verbatim}\vspace{-10pt}

\noindent \textbf{Execution}: A command-line launching of the model experiment (10000 years integration) would be:
\vspace{-10pt}\begin{verbatim}./runcgenie.sh cgenie_eb_go_gs_ac_bg_itfclsd_16l_JH_BASE /
EXAMPLE_worjh2_PO4_SPIN 10000\end{verbatim}\vspace{-5pt}

\noindent \textbf{Relevant HOW-TO}: 


%---------------------------------------------------------------------------------------------------------------------------------

\subsection{Modern 36x36x16 configuration; preindustrial [ALTERNATIVE]}\label{EXAMPLE_worjh2_PO4_PREINDUST}

This \textit{spin-up} is as per \texttt{EXAMPLE\_worjh2\_PO4\_SPIN} except it is configured with 'full' anthropogenic tracers.

\noindent \textbf{Physics configuration}: GOLDSTEIN ocean + sea-ice + EMBM atmosphere modules. Climatology is seasonal and identical to that described in \textit{Cao et al.} [2009] (and references therein).

\noindent \textbf{Biogeochemistry configuration}: Basic ocean (and atmosphere) carbon cycle as described \textit{Cao et al.} [2009], PLUS radiocarbon and CFC tracers. Atmospheric restoring of CO2 (plus d13C and d14C) and CFCs.

\noindent \textbf{Base-config} The \textit{base-config} file is named:
\vspace{-10pt}\begin{verbatim}cgenie_eb_go_gs_ac_bg_itfclsd_16l_JH_ANTH\end{verbatim}\vspace{-10pt}
and defines the use (and initial values) of the following tracers (in addition to those described for \texttt{cgenie\_eb\_go\_gs\_ac\_bg\_itfclsd\_16l\_JH\_BASE})\footnote{Before using this new \textit{base-config} for the first time, you will need to do a \texttt{make cleanall}.}:
\begin{compactenum}
	
	\item Atmospheric (gaseous) tracers (\texttt{gm\_atm\_select\_xx}):
	\\\texttt{ia\_pCO2\_14C} (xx=5), \texttt{ia\_pCFC11} (xx=18), \texttt{ia\_pCFC11} (xx=19)
	\item Ocean (dissolved) tracers (\texttt{gm\_ocn\_select\_xx}):
	\\\texttt{io\_DIC\_14C} (xx=5), \texttt{io\_DOM\_C\_14C} (xx=17), \texttt{io\_CFC11} (xx=45), \texttt{io\_CFC12} (xx=46)
		
\end{compactenum}
By default, a zero concentration for CFCs (in ocean and atmosphere) is set, while the d14C isotopic composition of all carbon species is set to 0 per mil.

\noindent \textbf{User-config} The \textit{user-config} file is named:
\vspace{-10pt}\begin{verbatim}EXAMPLE_worjh2_PO4_PREINDUST\end{verbatim}\vspace{-10pt}
and differs from the equivalent standard modern configuration in:
\begin{compactitem}
	
	\item \texttt{--- FORCINGS ---}
	\\ The \textit{forcing} prescribes fixed boundary conditions of atmospheric pCO2 and plus its isotopes and CFCs.
	The parameter values that follow simply scale atmospheric composition:
	\vspace{-5pt}\begin{verbatim}
	bg_ctrl_force_oldformat=.false.
	bg_par_forcing_name="worjh2_preindustrial"
	\end{verbatim}\vspace{-5pt}

\end{compactitem}

\noindent \textbf{Execution}: A command-line launching of the model experiment (10000 years integration) would be:
\vspace{-10pt}\begin{verbatim}./runcgenie.sh cgenie_eb_go_gs_ac_bg_itfclsd_16l_JH_ANTH /
EXAMPLE_worjh2_PO4_PREINDUST 10000\end{verbatim}\vspace{-5pt}

\noindent \textbf{Relevant HOW-TO}:


%---------------------------------------------------------------------------------------------------------------------------------

\subsection{Modern 36x36x16 configuration with an iron cycle}\label{EXAMPLE_worjh2_PO4Fe_SPIN}

This \textit{spin-up} is as per \texttt{EXAMPLE\_worjh2\_PO4\_SPIN} except it is configured with an Fe cycle.

\noindent \textbf{Physics configuration}: GOLDSTEIN ocean + sea-ice + EMBM atmosphere modules. Climatology is seasonal and identical to that described in \textit{Cao et al.} [2009] (and references therein).

\noindent \textbf{Biogeochemistry configuration}: The ocean carbon cycle includes an iron cycle and co-limitation of biological productivity and is as described in \textit{Ridgwell and De'Ath} [in prep]. During the spin-up, the ocean is forced into equilibrium with Preindustrial atmospheric concentrations of: CO2 and O2, plus the d13C of CO2, via a restoring \textit{forcing} of atmospheric composition.

\noindent \textbf{Base-config} The \textit{base-config} file is named:
\vspace{-10pt}\begin{verbatim}cgenie_eb_go_gs_ac_bg_itfclsd_16l_JH_BASEFe\end{verbatim}\vspace{-10pt} which defines the use (and initial values) of the following tracers\footnote{See the \texttt{cGENIE} \textit{Namelist} table for a description of the tracer numbering scheme.}:

\begin{compactenum}
	\item Atmospheric (gaseous) tracers (\texttt{gm\_atm\_select\_xx}):
	\\\texttt{ia\_pCO2} (xx=3), \texttt{ia\_pCO2\_13C} (xx=4), \texttt{ia\_pO2} (xx=6)
		(in addition to atmospheric temperature and humidity)
	\item Ocean (dissolved) tracers (\texttt{gm\_ocn\_select\_xx}):
	\\\texttt{io\_DIC} (xx=3), \texttt{io\_DIC\_13C} (xx=4), \texttt{io\_PO4} (xx=8), \texttt{io\_Fe} (xx=9), \texttt{io\_O2} (xx=10), \texttt{io\_ALK} (xx=12), io\_DOM\_C (xx=15), \texttt{io\_DOM\_C\_13C} (xx=16), \texttt{io\_DOM\_P} (xx=20), \texttt{io\_DOM\_Fe} (xx=22), \texttt{io\_FeL} (xx=23), \texttt{io\_L} (xx=24)
		(in addition to ocean temperature and salinity)
	\item The corresponding sedimentary (solid) tracers (\texttt{gm\_sed\_select\_xx}) are also selected:
	\\\texttt{is\_POC} (xx=3), \texttt{is\_POC\_13C} (xx=4), \texttt{is\_POP} (xx=8), \texttt{is\_POFe} (xx=10), \texttt{is\_POM\_Fe} (xx=13), \texttt{is\_CaCO3} (xx=14), \texttt{is\_CaCO3\_13C} (xx=15), \texttt{is\_CaCO3\_Fe} (xx=21), \texttt{is\_det} (xx=22), \texttt{is\_det\_Fe} (xx=25), \texttt{is\_ash} (xx=32), \texttt{is\_POC\_frac2} (xx=33), \texttt{is\_CaCO3\_frac2} (xx=34), \texttt{is\_CaCO3\_age} (xx=36)
\end{compactenum}


\noindent \textbf{User-config} The \textit{user-config} file is named:
\vspace{-10pt}\begin{verbatim}EXAMPLE_worjh2_PO4Fe_SPIN\end{verbatim}\vspace{-10pt} and contains the following parameter specifications\footnote{Mostly (but not always) these represent changes from the default and thus it would be possible to conduct an identical experiment with slightly fewer namelist specification. Some of the (mainly biological) namelist values are re-defined (identically) for completeness.}:

\begin{compactitem}
	
	\item \texttt{--- BIOLOGICAL NEW PRODUCTION ---}
	\\ \texttt{bg\_par\_bio\_prodopt='bio\_PFe'} == sets the P+Fe nutrient co-limitation 'biological' scheme. See: \textit{Ridgwell and De'Ath} [in prep] for a description of this (plus the other 3 lister parameters and their values).
	
	\item \texttt{--- ORGANIC MATTER EXPORT RATIOS ---}
	\\ Parameters as described in \textit{Ridgwell and De'Ath} [in prep].
	
	\item \texttt{--- INORGANIC MATTER EXPORT RATIOS ---}
	\\ Parameters as defined in \textit{Cao et al.} [2009] and based on the parameterization described in \textit{Ridgwell et al.} [2007a,b].
	
	\item \texttt{--- REMINERALIZATION ---}
	\\ Parameters mostly as defined in \textit{Cao et al.} [2009] and based on the parameterizations described in \textit{Ridgwell et al.} [2007a]., except:
	\\ The lifetime of DOM (\texttt{bg\_par\_bio\_remin\_DOMlifetime}), and 'initial fractional abundance of POC component' (\texttt{bg\_par\_bio\_remin\_POC\_frac2}) adopt parameter values as described in \textit{Ridgwell and De'Ath} [in prep].	
	
	\item \texttt{--- IRON ---}
	\\ Sets the Fe cycle, including:
	\begin{compactitem}
	\item	aeolian Fe solubility (\texttt{bg\_par\_det\_Fe\_sol})
		\item scavenging (\texttt{bg\_par\_scav\_Fe\_sf\_POC})
		\end{compactitem}
		\noindent See: \textit{Ridgwell and De'Ath} [in prep].
	
	\item \texttt{--- FORCINGS ---}
	\\ Firstly specifies the 'new' \textit{forcing} description syntax (\texttt{bg\_ctrl\_force\_oldformat=.false.}). The actual forcing applied is specified as \texttt{worjh2\_RpCO2\_Rp13CO2\_FeMahowald2006} (the files of which live in the equivalently named subdirectory of \texttt{\~{}/cgenie/genie-forcings}). The \textit{forcing} prescribes fixed boundary conditions of atmospheric pCO2 and d13C, plus a dust flux following \textit{Mahowald et al.} [2006] and consists of:
	\begin{compactitem}
		\item Selection of forcings:
		\begin{compactenum}
			\item  \texttt{configure\_forcings\_atm.dat} == Selection of restoring forcing\footnote{Time-constant for all \textit{restorings} set to 0.1 years.} of:
			\\\texttt{ia\_pCO2}, \texttt{ia\_pCO2\_13C}
			\item  \texttt{configure\_forcings\_ocn.dat} == No ocean tracer forcings.
			\item  \texttt{configure\_forcings\_sed.dat} == Selection of a flux forcing of:
			\\\texttt{is\_det}
		\end{compactenum}
		\item Spatial and temporal definition of forcings. All three selected forcings have a file containing time-dependent information associated with them\footnote{See: \textit{User manual}.}: \texttt{biogem\_force\_restore\_yyy\_xxx\_sig.dat}. In addition, the dust flux forcing has a 2D spatial pattern associated with it:
		\\ \texttt{biogem\_force\_flux\_sed\_det\_SUR.dat}.
		The parameter values at the end of this section simply scale atmospheric composition:
		\vspace{-5pt}\begin{verbatim}
		bg_par_atm_force_scale_val_3=278.0E-06
		bg_par_atm_force_scale_val_4=-6.5
		\end{verbatim}\vspace{-5pt}
	\end{compactitem}
		
	\item \texttt{--- MISC ---}
	\\ Finally: \textit{tracer auditing} is turned off and a closed (ocean+atmosphere) system carbon cycle (actually the default) is specified.
	
\end{compactitem}

\noindent \textbf{Execution}: A command-line launching of the model experiment (10000 years integration) would be:
\vspace{-10pt}\begin{verbatim}./runcgenie.sh cgenie_eb_go_gs_ac_bg_itfclsd_16l_JH_BASEFe /
EXAMPLE_worjh2_PO4Fe_SPIN 10000\end{verbatim}\vspace{-5pt}

\noindent \textbf{Relevant HOW-TO}: 


%---------------------------------------------------------------------------------------------------------------------------------

\subsection{Modern 36x36x16 configuration + Fe \& CH4 cycles}\label{EXAMPLE_worjh2_PO4Fe_CH4_SPIN}

This \textit{spin-up} is configured as per \texttt{EXAMPLE\_worjh2\_PO4Fe\_SPIN} except it has an added CH4 cycle.

\noindent \textbf{Physics configuration}: GOLDSTEIN ocean + sea-ice + EMBM atmosphere modules. Climatology is seasonal and identical to that described in \textit{Cao et al.} [2009] (and references therein).

\noindent \textbf{Biogeochemistry configuration}: Basic ocean (and atmosphere) carbon cycle as described \textit{Cao et al.} [2009]. Atmospheric restoring of CO2 and CH4 (plus d13C of both).

\noindent \textbf{Base-config} The \textit{base-config} file is named:
\vspace{-10pt}\begin{verbatim}cgenie_eb_go_gs_ac_bg_itfclsd_16l_JH_BASEFeCH4\end{verbatim}\vspace{-10pt}
and defines the use (and initial values) of the following tracers (in addition to those described for \texttt{cgenie\_eb\_go\_gs\_ac\_bg\_itfclsd\_16l\_JH\_BASEFe})\footnote{Before using this new \textit{base-config} for the first time, you will need to do a \texttt{make cleanall}.}:
\begin{compactenum}
	
	\item Atmospheric (gaseous) tracers (\texttt{gm\_atm\_select\_xx}):
	\\\texttt{ia\_pCH4} (xx=10), \texttt{ia\_pCH4\_13C} (xx=11)
	\item Ocean (dissolved) tracers (\texttt{gm\_ocn\_select\_xx}):
	\\\texttt{io\_CH4} (xx=25), \texttt{io\_CH4\_13C} (xx=26)
		
\end{compactenum}
By default, a zero concentration of CH4 (in ocean and atmosphere) are set, while the isotopic composition of both pCH4 (atmosphere) and CH4 (ocean, dissolved) is set to -60 per mil.

\noindent \textbf{User-config} The \textit{user-config} file is named:
\vspace{-10pt}\begin{verbatim}EXAMPLE_worjh2_PO4Fe_CH4_SPIN\end{verbatim}\vspace{-10pt}
and differs from the equivalent standard modern configuration in:
\begin{compactitem}
	
	\item \texttt{--- FORCINGS ---}
	\\ The \textit{forcing} prescribes fixed boundary conditions of atmospheric pCO2 and d13C, PLUS fixed boundary conditions of pCH4 and d13C (of CH4), in addition to a surface ocean dust flux.
	The parameter values that follow simply scale atmospheric composition:
	\vspace{-5pt}\begin{verbatim}
	bg_ctrl_force_oldformat=.false.
	bg_par_forcing_name="worjh2_RpCO2_Rp13CO2_RpCH4_Rp13CH4_FeMahowald2006"
	bg_par_atm_force_scale_val_3=278.0E-06
	bg_par_atm_force_scale_val_4=-6.5
	bg_par_atm_force_scale_val_10=1700.0E-9
	bg_par_atm_force_scale_val_11=-60.0
	\end{verbatim}\vspace{-5pt}
	Note that the atmospheric CH4 restoring concentration is specified here as modern (ca. 1700 ppb == 1700.0E-9 atm).
		
	\item \texttt{--- MISC ---}
	\\ Finally: an oxidation rate constant for CH4 in the ocean is prescribed:
	\vspace{-5pt}\begin{verbatim}bg_par_bio_remin_CH4rate=0.00004\end{verbatim}\vspace{-5pt}
	and has units of d-1.\footnote{Note that this particular value does not necessarily reflect any ocean reality ...}

\end{compactitem}

\noindent \textbf{Execution}: A command-line launching of the model experiment (10000 years integration) would be:
\vspace{-10pt}\begin{verbatim}./runcgenie.sh cgenie_eb_go_gs_ac_bg_itfclsd_16l_JH_BASEFeCH4 /
EXAMPLE_worjh2_PO4Fe_CH4_SPIN 10000\end{verbatim}\vspace{-5pt}

\noindent \textbf{Relevant HOW-TO}: 


%---------------------------------------------------------------------------------------------------------------------------------

\subsection{Eocene 36x36x16 configuration + CH4 cycle}\label{EXAMPLE_p0055c_PO4_CH4_SPIN}

This example uses an early Eocene continental configuration, with a basic (P-only) based ocean carbon cycle but with global biogeochemical cycling of CH4 included.

\noindent \textbf{Physics configuration}: GOLDSTEIN ocean + sea-ice + EMBM atmosphere modules. Adjusted planetary albedo and solar constant. Adjusted continental configuration. Forcing with seasonal insolation (but annual averaged wind stress and winds). See: \textit{Ridgwell and Schmidt} [2010].

\noindent \textbf{Biogeochemistry configuration}: Basic ocean (and atmosphere) carbon cycle as described \textit{Cao et al.} [2009] but with modifications following \textit{Ridgwell and Schmidt} [2010] (and described below). Atmospheric restoring of CO2 and CH4 (plus d13C of both).

\noindent \textbf{Base-config} The \textit{base-config} file is named:
\vspace{-10pt}\begin{verbatim}cgenie_eb_go_gs_ac_bg_hadcm3l_eocene_36x36x16_2i_080928_BASECH4 \end{verbatim}\vspace{-10pt}
and differs from the equivalent standard modern configuration in:
	\begin{compactitem}
	\item An early Eocene continental configuration is prescribed, and the grid started at -180E.
		\item CH4 (and d13C of CH4) tracers are selected as additional tracers. 
		\item Ocean temperatures are initialized at 10C:
		\\ \texttt{go\_10=10.}0, \texttt{go\_10=10.0}.
		\item Solar constant reduced by 0.46\% for end Paleocene:
		\\ \texttt{ma\_genie\_solar\_constant=1361.7}.
		\item Planetary albedo adjusted:
		\vspace{-5pt}\begin{verbatim}
ea_albedop_offs=0.200
ea_albedop_amp=0.260
ea_albedop_skew=0.0
ea_albedop_skewp=0
ea_albedop_mod2=-0.000
ea_albedop_mod4=0.000
ea_albedop_mod6=0.250
\end{verbatim}\vspace{-5pt}
\item Ocean salinity reduced by 1 per mil to take into account absence of large land-based ice sheets:
\\ \texttt{go\_saln0=33.9.}
	\end{compactitem}

\noindent \textbf{User-config} The \textit{user-config} file is named:
\vspace{-10pt}\begin{verbatim} EXAMPLE_p0055c_PO4_CH4_SPIN \end{verbatim}\vspace{-10pt}
and differs from the equivalent standard modern configuration in:
\begin{compactitem}
		\item \texttt{--- INORGANIC MATTER EXPORT RATIOS ---}
		\\ A uniform CaCO3:POC biological export ratio is set:
\vspace{-5pt}\begin{verbatim}bg_par_bio_red_POC_CaCO3=0.150\end{verbatim}\vspace{-5pt}
and made independent of ambient saturation state by:
\vspace{-5pt}\begin{verbatim}bg_par_bio_red_POC_CaCO3_pP=0.0\end{verbatim}\vspace{-5pt}
	\item \texttt{--- REMINERALIZATION ---}
	\\ An oxidation rate constant for CH4 in the ocean is prescribed:
\vspace{-5pt}\begin{verbatim}bg_par_bio_remin_CH4rate=0.00004\end{verbatim}\vspace{-5pt}
and has units of d-1.\footnote{Note that this particular value does not necessarily reflect any ocean reality ...}
		\item \texttt{--- FORCINGS ---}
	\\ The selected \textit{forcing} prescribes fixed boundary conditions of atmospheric pCO2 and d13C, PLUS pCH4 and d13C (of CH4):
\vspace{-5pt}\begin{verbatim}bg_par_forcing_name="pyyyyz_RpCO2_Rp13CO2_RpCH4_Rp13CH4"\end{verbatim}\vspace{-5pt}
	The normalized (unit) values contained in the forcing are then scaled:
	\vspace{-5pt}\begin{verbatim}
bg_par_atm_force_scale_val_3=834.0E-06
bg_par_atm_force_scale_val_4=-4.9
bg_par_atm_force_scale_val_10=3500.0E-9
bg_par_atm_force_scale_val_11=-60.0
		\end{verbatim}\vspace{-5pt}
to give x3 CO2 and approximately x5 CH4.
			\\ A (simulated) early Eocene wind field is specified for the calculation of air-sea gas exchange:
\vspace{-5pt}\begin{verbatim}bg_par_windspeed_file="p0055c_windspeed.dat"\end{verbatim}\vspace{-5pt}
and the gas exchange coefficient is adjusted to give ~0.058 mol m-2 yr-1 uatm-1 global mean air-sea coefficient:
\vspace{-5pt}\begin{verbatim}bg_par_gastransfer_a=0.5196\end{verbatim}\vspace{-5pt}	
	\item \texttt{--- MISC ---}
	\\ Feedback between atmospheric greenhouse gas concentrations (implicitly: CH4 in addition to CO2) and climate is set:
\vspace{-5pt}\begin{verbatim}ea_36=y\end{verbatim}\vspace{-5pt}
	\end{compactitem}

\noindent \textbf{Execution}: A command-line launching of the model experiment (10000 years integration) would be:
\vspace{-11pt}\begin{verbatim}./runcgenie.sh cgenie_eb_go_gs_ac_bg_hadcm3l_eocene_36x36x16_2i_080928_BASECH4 /
EXAMPLE_p0055c_PO4_CH4_SPIN 10000\end{verbatim}\vspace{-5pt}


%---------------------------------------------------------------------------------------------------------------------------------

\subsection{Eocene 36x36x16 configuration + CH4 cycle [ALTERNATIVE]}\label{EXAMPLE_p0055c_PO4_CH4_SPIN2}

This example uses an early Eocene continental configuration, with a basic (P-only) based ocean carbon cycle. The global CH4 biogeochemical cycle is configured without atmospheric restoring.

\noindent \textbf{Physics configuration}: GOLDSTEIN ocean + sea-ice + EMBM atmosphere modules. Adjusted planetary albedo and solar constant. Adjusted continental configuration. Forcing with seasonal insolation (but annual averaged wind stress and winds). See: \textit{Ridgwell and Schmidt} [2010].

\noindent \textbf{Biogeochemistry configuration}: Basic ocean (and atmosphere) carbon cycle as described \textit{Cao et al.} [2009] but with modifications following \textit{Ridgwell and Schmidt} [2010] (and described below). Atmospheric restoring of CO2 (+ d13C). Prescribed 'wetland' flux to the atmosphere of CH4 (+ d13C).

\noindent \textbf{Base-config} The \textit{base-config} file is:
\vspace{-10pt}\begin{verbatim}cgenie_eb_go_gs_ac_bg_hadcm3l_eocene_36x36x16_2i_080928_BASECH4 \end{verbatim}\vspace{-10pt}
and is as per described in the example \texttt{EXAMPLE\_p0055c\_PO4\_CH4\_SPIN} (above).

\noindent \textbf{User-config} The \textit{used-config} file:
\vspace{-10pt}\begin{verbatim} EXAMPLE_p0055c_PO4_CH4_SPIN2 \end{verbatim}\vspace{-10pt}
differs from the example \texttt{EXAMPLE\_p0055c\_PO4\_CH4\_SPIN}:
\begin{compactitem}
		\item \texttt{--- FORCINGS ---}
	\\ The selected \textit{forcing} prescribes fixed boundary conditions only of atmospheric pCO2 (+ d13C):
\vspace{-5pt}\begin{verbatim}bg_par_forcing_name="pyyyyz_RpCO2_Rp13CO2"\end{verbatim}\vspace{-5pt}
with the normalized (unit) values contained in the forcing scaled as per \texttt{EXAMPLE\_p0055c\_PO4\_CH4\_SPIN}:
	\vspace{-5pt}\begin{verbatim}
bg_par_atm_force_scale_val_3=834.0E-06
bg_par_atm_force_scale_val_4=-4.9
		\end{verbatim}\vspace{-5pt}
			A steady flux of CH4 (+ 13C) to the atmosphere is prescribed (as if from wetlands etc.):
	\vspace{-5pt}\begin{verbatim}
ac_par_atm_wetlands_FCH4=0.6206165E+14
ac_par_atm_wetlands_FCH4_d13C=-60.0
		\end{verbatim}\vspace{-5pt}
Refer to the \textit{HOW-TO} for details of how this value is determined.
	\end{compactitem}

\noindent \textbf{Execution}: Command-line launching of the model experiment for a 10000 year integration:
\vspace{-10pt}\begin{verbatim}./runcgenie.sh cgenie_eb_go_gs_ac_bg_hadcm3l_eocene_36x36x16_2i_080928_BASECH4 /
EXAMPLE_p0055c_PO4_CH4_SPIN2 10000\end{verbatim}\vspace{-5pt}

\noindent \textbf{Relevant HOW-TO}: 'Determine the CH4 flux required to achieve a particular atmospheric pCH4 value'


%---------------------------------------------------------------------------------------------------------------------------------

\subsection{Eocene 36x36x16 configuration + CH4 cycle + CLOSED CaCO3 weathering-sediment cycle}\label{EXAMPLE_p0055c_PO4_CH4_S72x72_SPIN}

This example uses an early Eocene continental configuration, with a basic (P-only) based ocean carbon cycle, and global CH4 biogeochemical cycling as before, but now with deep-sea (CaCO3) sedimentation and burial and weathering input in a '\textit{closed system}'.

\noindent \textbf{Physics configuration}: GOLDSTEIN ocean + sea-ice + EMBM atmosphere modules with with seasonal insolation forcing. Adjusted: continental configuration, planetary albedo, solar constant, ocean salinity, annual averaged wind stress and winds.

\noindent \textbf{Biogeochemistry configuration}: Basic ocean (and atmosphere) carbon cycle as described \textit{Cao et al.} [2009]. Atmospheric restoring of pCO2 (+ d13C) and of pCH4 (+ d13C).

\noindent \textbf{Base-config} The \textit{base-config} file is:
\vspace{-10pt}\begin{verbatim}cgenie_eb_go_gs_ac_bg_sg_rg_hadcm3l_eocene_36x36x16_2i_080928_BASECH4 \end{verbatim}\vspace{-10pt}
and is as per described in the example \texttt{EXAMPLE\_p0055c\_PO4\_CH4\_SPIN} (above).

\noindent \textbf{User-config} 
\\ This \textit{user-config} contains:
\vspace{-10pt}\begin{verbatim} EXAMPLE_p0055c_PO4_CH4_S72x72_SPIN \end{verbatim}\vspace{-10pt}
differs from the example \texttt{EXAMPLE\_p0055c\_PO4\_CH4\_SPIN}:
\begin{compactitem}
		\item \texttt{--- SEDIMENTS ---}
		\\ Bioturbation of the surface sediments turned 'off':
\vspace{-5pt}\begin{verbatim}
sg_ctrl_sed_bioturb=.false.
		\end{verbatim}\vspace{-5pt}
		\item \texttt{--- FORCINGS ---}
	\\ The selected \textit{forcing} prescribes fixed boundary conditions only of atmospheric pCO2 (+ d13C) plus pCH4 (+d13C):
\vspace{-5pt}\begin{verbatim}
bg_par_forcing_name="p0055c_RpCO2_Rp13CO2_RpCH4_Rp13CH4_detzeebeTT0"
\end{verbatim}\vspace{-5pt}
		\item \texttt{--- MISC ---}
		\\ Prescription of a \textit{closed system} (sedimentation balancing weathering input):
\vspace{-5pt}\begin{verbatim}
bg_ctrl_force_sed_closedsystem=.true.
		\end{verbatim}\vspace{-5pt}
Different (from modern) initial ocean alkalinity:
\vspace{-5pt}\begin{verbatim}
bg_ocn_init_12=2.075E-03
		\end{verbatim}\vspace{-5pt}
	\end{compactitem}

\noindent \textbf{Execution}: Command-line launching of the model experiment for a 20000 year integration:
\vspace{-5pt}\begin{verbatim}./runcgenie.sh cgenie_eb_go_gs_ac_bg_sg_rg_hadcm3l_eocene_36x36x16_2i_080928_BASECH4 /
EXAMPLE_p0055c_PO4_CH4_S72x72_SPIN 20000\end{verbatim}

\noindent \textbf{Relevant HOW-TO}: 'Spin-up the full marine carbon cycle including sediments'


%---------------------------------------------------------------------------------------------------------------------------------

\subsection{Eocene 36x36x16 configuration + CH4 cycle + OPEN CaCO3 weathering-sediment cycle}\label{EXAMPLE_p0055c_PO4_CH4_S72x72_SPIN2}

This example uses an early Eocene continental configuration, with a basic (P-only) based ocean carbon cycle, and global CH4 biogeochemical cycling as before, but now with deep-sea (CaCO3) sedimentation and burial and weathering input in an '\textit{open system}'.

\noindent \textbf{Physics configuration}: GOLDSTEIN ocean + sea-ice + EMBM atmosphere modules with with seasonal insolation forcing. Adjusted: continental configuration, planetary albedo, solar constant, ocean salinity, annual averaged wind stress and winds.

\noindent \textbf{Biogeochemistry configuration}: Basic ocean (and atmosphere) carbon cycle as described \textit{Cao et al.} [2009]. Atmospheric restoring of pCO2 (+ d13C) plus prescribed 'wetland' flux to the atmosphere of CH4 (+ d13C).\footnote{Refer to the \textit{HOW-TO} for details of how to set the value of 'wetland' CH4 emissions.}

\noindent \textbf{Base-config} The \textit{base-config} file is:
\vspace{-10pt}\begin{verbatim}cgenie_eb_go_gs_ac_bg_sg_rg_hadcm3l_eocene_36x36x16_2i_080928_BASECH4 \end{verbatim}\vspace{-10pt}
and is as per described in the example \texttt{EXAMPLE\_p0055c\_PO4\_CH4\_SPIN} (above).

\noindent \textbf{User-config} 
\\ This \textit{user-config} contains:
\vspace{-10pt}\begin{verbatim} EXAMPLE_p0055c_PO4_CH4_S72x72_SPIN2 \end{verbatim}\vspace{-10pt}
differs from the example \texttt{EXAMPLE\_p0055c\_PO4\_CH4\_SPIN}:
\begin{compactitem}
		\item \texttt{--- SEDIMENTS ---}
		\\ Bioturbation of the surface sediments now turned 'on':
\vspace{-5pt}\begin{verbatim}
sg_ctrl_sed_bioturb=.true.
		\end{verbatim}\vspace{-5pt}
		\item \texttt{--- FORCINGS ---}
	\\ The selected \textit{forcing} prescribes fixed boundary conditions only of atmospheric pCO2 (+ d13C) only:
	\vspace{-5pt}\begin{verbatim}
bg_par_forcing_name="p0055c_RpCO2_Rp13CO2_detzeebeTT0"
\end{verbatim}\vspace{-5pt}
		\item \texttt{--- MISC ---}
		\\ Prescription of an \textit{open system}:
\vspace{-5pt}\begin{verbatim}
bg_ctrl_force_sed_closedsystem=.false.
		\end{verbatim}\vspace{-5pt}
	\end{compactitem}

\noindent \textbf{Execution}: Command-line launching of the model experiment for a 50000 year integration:
\vspace{-5pt}\begin{verbatim}./runcgenie.sh cgenie_eb_go_gs_ac_bg_sg_rg_hadcm3l_eocene_36x36x16_2i_080928_BASECH4 /
EXAMPLE_p0055c_PO4_CH4_S72x72_SPIN2 50000 EXAMPLE_p0055c_PO4_CH4_S72x72_SPIN\end{verbatim}

\noindent \textbf{Relevant HOW-TO}: 'Spin-up the full marine carbon cycle including sediments',
\\'Determine the CH4 flux required to achieve a particular atmospheric pCH4 value'


%---------------------------------------------------------------------------------------------------------------------------------
%--- cGENIE EXPERIMENTS ----------------------------------------------------------------------------------------------------------
%---------------------------------------------------------------------------------------------------------------------------------

\newpage
\section{Example Experiments}\label{Example Experiments}

A variety of different model experiment for reference and for use as a helpful starting-point (template) in creating model experiments.\footnote{Remember: when trying different examples -- the first time that a different \textit{base-config} is used, a \texttt{make cleanall} must be done.}


%---------------------------------------------------------------------------------------------------------------------------------

\subsection{Prescribed emission of CO2 into the atmosphere}\label{EXAMPLE_worjh2_PO4Fe_CO2EMISSIONS}

This experiment contains an example emission of CO2 (uniformly) to the atmosphere.

\noindent \textbf{User-config}: \texttt{EXAMPLE\_worjh2\_PO4Fe\_CO2EMISSION}
\\ This \textit{user-config} contains:

\begin{compactitem}
	\item A prescribed \texttt{forcing}: \texttt{worjh2\_FpCO2\_Fp13CO2\_FeMahowald2006}, which is configured in the \textit{user-config} as follows:
	\begin{compactenum}
		\item 
		\begin{verbatim}
		bg_par_atm_force_scale_val_03=0.0833e15
		bg_par_atm_force_scale_val_04=-27.0
		\end{verbatim}
		which scale the (unit) emissions as mol per year (0.0833e15 = 1 PgC) and the isotopic composition of the emissions, respectively, and
		\item 
		\begin{verbatim}
		bg_par_atm_force_scale_time_03=1.0E1
		bg_par_atm_force_scale_time_04=1.0E1
		\end{verbatim}
		which scale\footnote{Equal scaling of both tracers must be done.} the duration of a (unit) pulse of emissions which in this example is 10 (1.0x10\^1) years.
	\end{compactenum}
\end{compactitem}

\noindent \textbf{Base-config}: \texttt{genie\_eb\_go\_gs\_ac\_bg\_itfclsd\_16l\_JH\_BASEFe}

\noindent \textbf{Pre-requisites}: A spin-up such as \texttt{EXAMPLE\_worjh2\_PO4Fe\_SPIN}

\noindent \textbf{Execution}:
\vspace{-5pt}\begin{verbatim}
./runcgenie.sh cgenie_eb_go_gs_ac_bg_itfclsd_16l_JH_BASEFe / 
EXAMPLE_worjh2_PO4Fe_CO2EMISSION 100 EXAMPLE_worjh2_PO4Fe_SPIN
\end{verbatim}\vspace{-5pt}

\noindent \textbf{Ideas for further development}: --

\noindent \textbf{Relevant HOW-TO}: --


%---------------------------------------------------------------------------------------------------------------------------------

\subsection{Prescribed emission of CH4 into the atmosphere}\label{EXAMPLE_worjh2_PO4Fe_CH4EMISSION}

This experiment contains an example emission of CH4 (uniformly) to the atmosphere.

\noindent \textbf{User-config}: \texttt{EXAMPLE\_worjh2\_PO4Fe\_CH4EMISSION}
\\ This \textit{user-config} contains:
\begin{compactitem}
	\item A prescribed \texttt{forcing}: \texttt{worjh2\_FpCH4\_Fp13CH4\_FeMahowald2006}, which is configured in the \textit{user-config} as follows:
\begin{compactenum}
	\item 
	\begin{verbatim}
bg_par_atm_force_scale_val_10=0.0833e15
bg_par_atm_force_scale_val_11=-27.0
		\end{verbatim}
		which scale the (unit) emissions as mol per year (0.0833e15 = 1 PgC) and the isotopic composition of the emissions, respectively, and
	\item 
	\begin{verbatim}
bg_par_atm_force_scale_time_10=1.0E1
bg_par_atm_force_scale_time_11=1.0E1
		\end{verbatim}
		which scale the duration of a (unit) pulse of emissions.
\end{compactenum}
\end{compactitem}

\noindent \textbf{Base-config}: \texttt{genie\_eb\_go\_gs\_ac\_bg\_itfclsd\_16l\_JH\_BASEFeCH4}

\noindent \textbf{Pre-requisites}: A spin-up including a CH4 cycle, such as \texttt{EXAMPLE\_worjh2\_PO4Fe\_CH4\_SPIN}

\noindent \textbf{Execution}:
\vspace{-5pt}\begin{verbatim}
./runcgenie.sh cgenie_eb_go_gs_ac_bg_itfclsd_16l_JH_BASEFeCH4 / 
EXAMPLE_worjh2_PO4Fe_CH4EMISSION 100 EXAMPLE_worjh2_PO4Fe_CH4_SPIN
		\end{verbatim}\vspace{-5pt}

\noindent \textbf{Ideas for further development}:

\noindent \textbf{Relevant HOW-TO}: --


%---------------------------------------------------------------------------------------------------------------------------------

\subsection{Prescribed injection of DIC at a specific location in the ocean.}\label{EXAMPLE_worjh2_PO4Fe_DICINJECTION}

This experiment contains an example injection of dissolved inorganic carbon (DIC) at a specific point location  in the ocean.

\noindent \textbf{User-config}: \texttt{EXAMPLE\_worjh2\_PO4Fe\_DICINJECTION}
\\ This \textit{user-config} contains:
\begin{compactitem}
	\item A prescribed \texttt{forcing}: \texttt{worjh2\_FDIC\_F13DIC\_FeMahowald2006}, which is configured in the \textit{user-config} as follows:
\begin{compactenum}
	\item 
	\begin{verbatim}
bg_par_ocn_force_scale_val_03=0.0833e15
bg_par_ocn_force_scale_val_04=-27.0
		\end{verbatim}
		which scale the (unit) emissions as mol per year (0.0833e15 = 1 PgC) and the isotopic composition of the emissions, respectively,
	\item 
	\begin{verbatim}
bg_par_ocn_force_scale_time_03=1.0E1
bg_par_ocn_force_scale_time_04=1.0E1
		\end{verbatim}
		which scale the duration of a (unit) pulse of emissions\footnote{Equal scaling of both tracers must be done.}, and
	\item 
	\begin{verbatim}
bg_par_force_point_i=18
bg_par_force_point_j=26
bg_par_force_point_k=7
		\end{verbatim}
		which defines the location of a point source for the emissions\footnote{Note that a point location can instead be set in the \textit{forcing} itself}, which in this example is somewhere at the bottom of the Gulf of Mexico.
\end{compactenum}
\item The specification for the saving of additional 2D data fields for ocean bottom waters: 
\\ \texttt{bg\_ctrl\_data\_save\_slice\_ocnsed=.true.}.
\end{compactitem}

\noindent \textbf{Base-config}: \texttt{genie\_eb\_go\_gs\_ac\_bg\_itfclsd\_16l\_JH\_BASEFe}

\noindent \textbf{Pre-requisites}: A spin-up such as \texttt{EXAMPLE\_worjh2\_PO4Fe\_SPIN}

\noindent \textbf{Execution}:
\vspace{-5pt}\begin{verbatim}
./runcgenie.sh cgenie_eb_go_gs_ac_bg_itfclsd_16l_JH_BASEFe / 
EXAMPLE_worjh2_PO4Fe_DICINJECTION 100 EXAMPLE_worjh2_PO4Fe_SPIN
		\end{verbatim}\vspace{-5pt}

\noindent \textbf{Ideas for further development}:
\begin{compactenum}
	\item A trivial change to the experiment would be to set a different injection location (and/or rate and/or duration) ...
		\item Simple changes can also be made so that DIC is injected to the ocean as a whole or to the surface only (and uniformly). This requires modification of the \textit{forcing} but is relatively straight-forward. All this requires is a change to the file: \texttt{configure\_forcings\_ocn.dat}; '\texttt{COLUMN \#06}'.
				\item A pattern of DIC injection can also be prescribed: e.g., release at all bottom water locations everywhere, or all bottom-waters in a certain depth range and/or basin, or a surface flux with a specific patter (distribution). [\textbf{See \textit{HOW-TO}}]
\end{compactenum}

\noindent \textbf{Relevant HOW-TO}: ---


%---------------------------------------------------------------------------------------------------------------------------------

\subsection{Prescribed injection of (dissolved) CH4 at a specific location in the ocean.}\label{EXAMPLE_worjh2_PO4Fe_CH4INJECTION}

This experiment describes an example injection of (dissolved) CH4 at a point location in the ocean.

\noindent \textbf{User-config}: \texttt{EXAMPLE\_worjh2\_PO4Fe\_CH4INJECTION}
\\ This \textit{user-config} contains:
\begin{compactitem}
	\item A prescribed \texttt{forcing}: \texttt{worjh2\_FCH4\_F13CH4\_FeMahowald2006}, which is configured in the \textit{user-config} as follows:
\begin{compactenum}
	\item 
	\begin{verbatim}
bg_par_ocn_force_scale_val_25=0.0833e15
bg_par_ocn_force_scale_val_26=-60.0
		\end{verbatim}
		which scale the (unit) emissions as mol per year (0.0833e15 = 1 PgC) and the isotopic composition of the emissions, respectively,
	\item 
	\begin{verbatim}
bg_par_ocn_force_scale_time_25=1.0E1
bg_par_ocn_force_scale_time_26=1.0E1
		\end{verbatim}
		which scale the duration of a (unit) pulse of emissions\footnote{Equal scaling of both tracers must be done.}, and
	\item 
	\begin{verbatim}
bg_par_force_point_i=18
bg_par_force_point_j=26
bg_par_force_point_k=7
		\end{verbatim}
		which defines the location of a point source for the emissions\footnote{Note that a point location can instead be set in the \textit{forcing} itself}, which in this example is somewhere at the bottom of the Gulf of Mexico.
\end{compactenum}
\end{compactitem}

\noindent \textbf{Base-config}: \texttt{genie\_eb\_go\_gs\_ac\_bg\_itfclsd\_16l\_JH\_BASEFeCH4}

\noindent \textbf{Pre-requisites}: A spin-up including a CH4 cycle, such as \texttt{EXAMPLE\_worjh2\_PO4Fe\_CH4\_SPIN}

\noindent \textbf{Execution}:
\vspace{-10pt}\begin{verbatim}
./runcgenie.sh cgenie_eb_go_gs_ac_bg_itfclsd_16l_JH_BASEFeCH4 / 
EXAMPLE_worjh2_PO4Fe_CH4INJECTION 100 EXAMPLE_worjh2_PO4Fe_CH4_SPIN
		\end{verbatim}

\noindent \textbf{Further development ideas}:
\begin{compactenum}
	\item A trivial change to the experiment would be to set a different injection location (and/or rate and/or duration) ...
		\item Simple changes can also be made so that dissolved CH4 is injected to the ocean as a whole or to the surface only (and uniformly). This requires modification of the \textit{forcing} but is relatively straight-forward. All this requires is a change to the file:
		\\ \texttt{configure\_forcings\_ocn.dat}; '\texttt{COLUMN \#06}'.
				\item A pattern of CH4 injection can also be prescribed: e.g., release at all bottom water locations everywhere, or all bottom-waters in a certain depth range and/or basin, or a surface flux with a specific patter (distribution). [\textbf{See \textit{HOW-TO}}]
				\item Without a restoring CH4 value in the atmosphere as was specified in:
				\\\texttt{EXAMPLE\_worjh2\_PO4Fe\_CH4\_SPIN}
				\\means that the atmospheric CH4 concentration will quickly decay to zero (except in the case of massive prescribed CH4 injections, particularly at depths close to the ocean surface). Adding a an additional (restoring) forcing of a fixed CH4 concentration in the atmosphere will obviously prevent the full impact of CH4 injection in the ocean being simulated. Hence, an additional atmospheric CH4 emission source is required that balances (primarily) atmospheric oxidation to achieve an appropriate initial non-zero (e.g., pre-industrial or modern) concentration of CH4 in the atmosphere prior to injection. This requires a parameter defining a baseline flux of CH4 to the atmosphere to be set (and before that: diagnosed consistent with a steady-state CH4 concentration). [\textbf{See \textit{HOW-TO}}]
\end{compactenum}

\noindent \textbf{Relevant HOW-TO}: ---


%---------------------------------------------------------------------------------------------------------------------------------

\subsection{Prescribed emission of CH4 into the atmosphere (Eocene configuration)}\label{EXAMPLE_p0055c_PO4_CH4EMISSION}

This experiment contains an example emission of CH4 (uniformly) to the atmosphere and is designed as a template for adapting to injection of CH4 in the ocean, and/or emission of CO2 to the atmosphere and/or CO2 injection in the ocean.

\noindent \textbf{User-config}: \texttt{EXAMPLE\_p0055c\_PO4\_CH4EMISSION}
\\ This \textit{user-config} has the following notable features:
\begin{compactitem}
	\item The prescribed \textit{forcing}:
\vspace{-5pt}\begin{verbatim}pyyyyz_FpCO2_Fp13CO2_FpCH4_Fp13CH4\end{verbatim}\vspace{-5pt}
	is generic in that CH4 and/or CO2 can equally (and even simultaneously) emitted to the atmosphere. In this example, the setup is for CH4 emission to the atmosphere and no release prescribed for CO2:
	\begin{verbatim}
bg_par_atm_force_scale_val_03=0.0
bg_par_atm_force_scale_val_04=0.0
bg_par_atm_force_scale_time_03=0.0
bg_par_atm_force_scale_time_04=0.0
bg_par_atm_force_scale_val_10=0.0833e15
bg_par_atm_force_scale_val_11=-60.0
bg_par_atm_force_scale_time_10=1.0E1
bg_par_atm_force_scale_time_11=1.0E1
		\end{verbatim}
\end{compactitem}

\noindent \textbf{Base-config}: \texttt{cgenie\_eb\_go\_gs\_ac\_bg\_hadcm3l\_eocene\_36x36x16\_2i\_080928\_BASECH4}

\noindent \textbf{Pre-requisites}: An Eocene configuration spin-up including a CH4 cycle, such as: \\ \texttt{EXAMPLE\_p0055c\_PO4\_CH4\_SPIN}

\noindent \textbf{Execution}:
\vspace{-5pt}\begin{verbatim}
./runcgenie.sh cgenie_eb_go_gs_ac_bg_hadcm3l_eocene_36x36x16_2i_080928_BASECH4 / 
EXAMPLE_p0055c_PO4_CH4EMISSION 100 EXAMPLE_p0055c_PO4_CH4_SPIN
		\end{verbatim}

\noindent \textbf{Ideas for further development}:
\begin{compactenum}
\item Obviously: one modification is to replace CH4 release with CO2 release. (Or combine to create a simultaneous CO2+CH4 releases.)
\item The same \textit{user-config} can be modified for a CH4 (or CO2) injection into the ocean. For this, the generic forcing:
\vspace{-5pt}\begin{verbatim}pyyyyz_FDIC_F13DIC_FCH4_F13CH4\end{verbatim}\vspace{-5pt}
needs to be specified. The scaling factors for the corresponding CH4 injection\footnote{Note that an injection location must also be specified (as per e.g. \texttt{EXAMPLE\_worjh2\_PO4Fe\_CH4INJECTION} above).} would look like:
\begin{verbatim}
bg_par_atm_force_scale_val_03=0.0
bg_par_atm_force_scale_val_04=0.0
bg_par_atm_force_scale_time_03=0.0
bg_par_atm_force_scale_time_04=0.0
bg_par_atm_force_scale_val_10=0.0833e15
bg_par_atm_force_scale_val_11=-60.0
bg_par_atm_force_scale_time_10=1.0E1
bg_par_atm_force_scale_time_11=1.0E1
		\end{verbatim}
\end{compactenum}

\noindent \textbf{Relevant HOW-TO}: ---


%---------------------------------------------------------------------------------------------------------------------------------
%--- Contact Information ---------------------------------------------------------------------------------------------------------
%---------------------------------------------------------------------------------------------------------------------------------

\newpage
\section{Contact Information}

\begin{compactitem}
	\item Andy Ridgwell: \texttt{bandy@seao2.org}
\end{compactitem}



%=================================================================================================================================
%=== END DOCUMENT ================================================================================================================
%=================================================================================================================================

\end{document}

